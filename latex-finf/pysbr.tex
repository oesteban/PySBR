\documentclass{frontiers}
% From http://www.frontiersin.org/Neuroinformatics/authorguidelines, original research has:
% Abstract: < 2001 characters
% Running title: <6 words
% Figures & tables: <16
% Total length: <12.001 words
% Final PDF length: <13 pages

\usepackage[acronym,nowarn,nomain]{glossaries}
\usepackage{aliascnt}
\usepackage{relsize}
\usepackage[procnames]{listings}
\usepackage[font=footnotesize,labelfont=bf]{caption}
\usepackage[font=footnotesize]{subcaption}
\usepackage[framemethod=TikZ]{mdframed}
\usepackage{changebar}
\usepackage{url}

% Set this to true if you submit the article
%  it will put all the figures at the end 
\newboolean{submit}
\setboolean{submit}{false}

\ifthenelse{\boolean{submit}}
{
  % if this is the submission version put all images at the end
  % Should they be in the file at all? I think Frontiers wants the 
  % images as separate files 
  \usepackage[nofiglist,figuresonly]{endfloat}
  \renewcommand{\efloatseparator}{\mbox{}}
  %\newcommand{\insertgraphic}[2]{}
  \newcommand{\insertgraphic}[2]{\includegraphics[#1]{#2}}
  \newcommand{\fixme}[1]{}
}
{
  % Until we submit we can have fixmes, Setting the version to submit will 
  % let latex choke on this one forcing us to really fix it ;) 
  \newcommand{\fixme}[1]{{\color{red}{\bf FIXME: }\emph{#1}}}
  \newcommand{\insertgraphic}[2]{\includegraphics[#1]{#2}}
}

\usepackage[hidelinks]{hyperref}

\copyrightyear{2013}
\pubyear{}

%%% write here for which journal %%%
\def\journal{Neuroinformatics}
\def\DOI{}
\def\articleType{Original Research}
\def\keyFont{\fontsize{6}{11}\helveticabold }
\def\firstAuthorLast{Esteban {et~al}} %use et al only if is more than 1 author
\def\Authors{Oscar Esteban\,$^{1,3,*}$, 
             Gert Wollny\,$^{1,3}$, 
             Aida Ni\~nerola-Baiz\'an\,$^{2,3}$, 
             Berta Mart\'i-Fuster\,$^{2,3}$,
             Mar\'ia Jes\'us Ledesma-Carbayo\,$^{1,3}$, 
             Xavier Setoain\,$^{3,4}$, and 
             Andr\'es Santos\,$^{1,3}$}

\def\Address{$^{1}$ Biomedical Image Technologies (BIT), ETSI Telecomunicaci\'on, 
               Universidad Polit\'ecnica de Madrid, Madrid, Spain \\
             $^{2}$ Biophysics and Bioengineering Unit, Physiological Sciences Department I, 
               School of Medicine, University of Barcelona, Barcelona, Spain\\
             $^{3}$ Biomedical Research Networking center in Bioengineering, 
               Biomaterials and Nanomedicine (CIBER-BBN), Zaragoza, Spain\\
            $^{4}$ Servei de Medicina Nuclear, Hospital Cl\'inic, Barcelona, Spain}
\def\corrAuthor{O. Esteban}
\def\corrAddress{Biomedical Image Technologies (BIT), Av. Complutense 30, 
                 ETSI Telecomunicaci\'on, Lab. C203, E-28040 Madrid, Spain}
\def\corrEmail{oesteban@die.upm.es}


% Redefine maths and items -------------------------------------------------------------------
\newcommand{\emitem}[1]{\item \emph{#1}}
\newcommand{\vect}[1]{\ensuremath{\mathbf{#1}}}
\newcommand{\vx}{\vect{x}}
\newcommand{\vy}{\vect{y}}
\newcommand{\lm}[1]{\ensuremath{\mathbf{l}_{#1}}}
\newcommand{\ab}{\ensuremath{\partial \mathbf{A}}}
\newcommand{\gc}{\ensuremath{\mathbf{g}_{\ab}}}
\newcommand{\ma}{\ensuremath{\mu(\lm{1}, \ab)}}
\newcommand{\va}{\ensuremath{\sigma(\lm{1}, \ab)}}
\newcommand{\degree}{\ensuremath{^{\circ}}}
\newcommand{\pc}{\ensuremath{\mathbf{c}}}
\newcommand{\pb}{\ensuremath{\mathbf{b}}}
\newcommand{\pa}{\ensuremath{\mathbf{a}}}
\newcommand{\ps}{\ensuremath{\mathbf{s}}}
\newcommand{\ph}{\ensuremath{\mathbf{h}}}
\newcommand{\px}{\ensuremath{\mathbf{x}}}
\newcommand{\pr}{\ensuremath{\mathbf{r}}}
\newcommand{\hp}[1]{\ensuremath{\mathbf{h}_{#1}}}


% Change tt font size
\renewcommand*{\UrlFont}{\ttfamily\smaller\relax}

% Listings style -----------------------------------------------------------------------------
\definecolor{pink}{RGB}{255,0,90}
\definecolor{comments}{RGB}{50,120,110}
\definecolor{string}{RGB}{160,0,0}
\definecolor{keywords}{RGB}{0,150,0}
\lstset{
  language=Python,
  showstringspaces=false,
  formfeed=\newpage,
  tabsize=4,
  breaklines=true,
  basicstyle=\ttfamily\smaller\relax,
  keywordstyle=\color{keywords}\bfseries,
  commentstyle=\color{comments}\itshape,
  stringstyle=\color{string},
  showstringspaces=false,
  %identifierstyle=\color{green},
  procnamekeys={def,class},
  morekeywords={models, lambda, forms, as, from},
  numbers=left,
  numberstyle=\smaller\color{black!60},
  stepnumber=1,
  numbersep=5pt,
  frame=single,
  xleftmargin= 40pt,
  xrightmargin= 20pt,
  framexleftmargin=20pt,
  frameround=tttt,
  fillcolor=\color{gray!10},
  backgroundcolor=\color{gray!10}
}

% Algorithms style -----------------------------------------------------------------------------
\mdfdefinestyle{algstyle}{%
  backgroundcolor=gray!10,
  topline=false, rightline=false, leftline=false, bottomline=true,
  frametitlerule=true,
  skipabove=15pt, skipbelow=10pt,
  innertopmargin=\topskip,innerbottommargin=5pt,
  frametitlebackgroundcolor=white,
  frametitleaboveskip=10pt, frametitlebelowskip=0pt,
  frametitlerule=true, frametitlerulewidth=0.5pt,
  frametitlealignment=\raggedright,
  font=\footnotesize
  %subtitlebelowline=true
}


\newcounter{myalgorithm}
\newenvironment{myalgorithm}[1][]{%
  \begin{mdframed}[style=algstyle,frametitle={\refstepcounter{myalgorithm}Algorithm \themyalgorithm: #1}]%
}{%
  \end{mdframed}
}
\providecommand*{\myalgorithmautorefname}{Algorithm}

% Define acronyms ------------------------------------------------------------------------------
\newacronym[first={\emph{SPM} (Statistical Parametric Mapping, \cite{ashburner_unified_2005})}]%
{spm}{\emph{SPM}}{Statistical Parametric Mapping}
\newacronym[first={Symmetric Normalization (\emph{SyN}, \cite{avants_symmetric_2008})}]%
{syn}{\emph{SyN}}{Symmetric Normalization transformation model}
\newacronym[first={\emph{ANTS} (Advanced Normalization tools, \cite{avants_ants:_2013})}]%
{ants}{\emph{ANTS}}{Advanced Normalization tools}
\newacronym[first={\emph{ITK} (the Insight Registration \& Segmentation Toolkit, \url{http://www.itk.org})}]%
{itk}{\emph{ITK}}{Insight Registration \& Segmentation Toolkit}
\newacronym[first={\emph{PySBR} (shape-based registration in Python)}]%
{pysbr}{\emph{PySBR}}{Shape-based registration in Python}
\newacronym[first={\emph{nipype} (Neuroimaging in Python, Pipelines and Interfaces, \cite{gorgolewski_nipype:_2011})}]%
{nipype}{\emph{nipype}}{Neuroimaging in Python, Pipelines and Interfaces}
\newacronym[first={\emph{NeSt} (\url{www.brainnest.org}, \cite{ciarmiello_weighted_2013})}]%
{brainnest}{\emph{NeSt}}{BrainNest neuroimaging platform}
\newacronym[first={\emph{BasGan} \citep{calvini_basal_2007}}]%
{basgan}{\emph{BasGan}}{BrainNest neuroimaging platform}
\newacronym[first={intensity-based registration (which we will refer to as IBR in the following)}]%
{ibr}{IBR}{intensity-based registration}
\newacronym{mri}{MRI}{magnetic resonance imaging}
\newacronym{montecarlo}{MC}{Monte-Carlo}
\newacronym[longplural={regions of interest}]{roi}{ROI}{region of interest}
\newacronym{spect}{SPECT}{single photon emission computed tomography}
\newacronym{pet}{PET}{positron emission tomography}
\newacronym{mi}{MI}{mutual Information}
\newacronym{nmi}{NMI}{normalized mutual information}
\newacronym{dat}{DAT}{dopamine transporter}
\newacronym{datscan}{DaTSCAN}{dopamine transporter imaging}
\newacronym{t1}{T1w}{T1-weighted MRI}
\newacronym{fov}{FoV}{field of view}
\newacronym{pv}{PV}{partial volume effect}
\newacronym{ebs}{EBS}{elastic body splines}
\newacronym{pde}{PDE}{partial differential equation}
\newacronym{pca}{PCA}{principal components analysis}
\newacronym{lehr}{LEHR}{low-energy high-resolution}
\newacronym{csf}{CSF}{cerebrospinal fluid}
\newacronym{wm}{WM}{white matter}
\newacronym{gm}{GM}{grey matter}
\newacronym{json}{JSON}{JavaScript Object Notation}
\makeglossaries

\begin{document}
\onecolumn
\firstpage{1}

\title[Evaluating normalization]%
  {Evaluating spatial normalization in fully automated SPECT imaging quantification protocols}

\author[\firstAuthorLast ]{\Authors}
\address{}
\correspondance{}
\editor{}
\topic{Python in Neuroscience II}

\maketitle


\begin{abstract}
\section{}
Absolute quantification of \gls*{spect} will
  provide the necessary standardization for the diagnosis and monitoring
  of movement disorders and a number of related diseases.
In order to achieve this standardization, the full automation of the
  process requires precise image normalization to/from a standard space.
However, quantification protocols generally lack of a structural reference
  (e.g. \acrlong{t1}) for high-standard accuracy.
In this paper, we present a new evaluation framework implemented with \emph{nipype},
  and we propose \gls*{pysbr}, a new spatial normalization tool implemented in Python for 
  its use in the described environment.
\gls*{pysbr} detects particular shape-features in the SPECT images and
  minimizes a distance to the corresponding features in standard space.
Densification of the resulting sparse displacements field is achieved by using
  Elastic-body splines as deformation model.
We use the evaluation tool to study \gls*{pysbr} and a standard normalization
  scheme employing intensity-based nonlinear image registration, both starting 
  from an affine initialization. 
The evaluation was run on 60 synthetic  phantoms and 23 realistic datasets simulated 
  for a gamma-camera widely used in the clinical routine.
Finally, we compare the proposed tools on the normalization of 5 real datasets.
Results show that both approaches on average do not offer significant improvements over 
  the affine initialization.
Nevertheless, \gls*{pysbr} shows potential for the use in a different
  imaging settings with better signal-to-noise ratio that justifies its further development.


\tiny
  \section{Keywords:} nonlinear registration, spect, pet, quantification, movement disorders
\end{abstract}

%\twocolumn

\glsresetall[\acronymtype]

\section{Introduction}\label{sec:intro}

\subsection{Assessment of processing software}
\label{sec:intro_software}
Image processing is witnessing the proliferation of computational tools
  providing general and problem-specific solutions.
As a consequence, the number of evaluation and comparison papers is
  also rapidly increasing.
\cite{tustison_instrumentation_2013} describe that software should be considered 
 ``a type of instrument requiring proper calibration and maintenance for accurate measurements'', 
  and in order to avoid \emph{instrumentation bias} (\cite{Shattuck_validation_2009}), the proper planning 
  and the publication of such evaluations of brain image processing methods is imperative. 
We consider that the principles described in \cite{tustison_instrumentation_2013} are not only
  important for the design of evaluations and comparisons, 
  but that they can also be applied to the development of new image processing
  methods, i.e. in \emph{functional-test driven} strategies \citep{andrea_envisioning_2007}.
In this Research Topic, we show how rigorous evaluation pipelines can be designed
  with readily available software in Python \emph{before} the development of 
  problem-specific solutions .
We use a clinical application, the fully automated quantification of \gls*{datscan},
  as a vessel to illustrate how this approach can easily be implementable by using \gls*{nipype}.

\subsection{Clinical problem}\label{sec:intro_problem}
\Gls*{spect} imaging of the dopaminergic system using
  specific agents (e.g. $^{123}$I- or $^{99m}$Tc- based) is of great
  value to discriminate parkinsonian syndromes from other movement
  disorders by quantifying the \acrlong*{dat} binding \citep{bajaj_clinical_2013}.
Quantification of \gls*{datscan} can facilitate early 
  diagnosis, progression follow-up, and assessment of treatment
  strategies. 
In clinical routine, this application is generally
  performed by visual inspection of regions corresponding to the
  striatal structures in the \gls*{datscan}  (see \autoref{fig:01}, subfigure A).
Specifically, due to the predominant uptake that occurs at caudal and putaminal areas,
  \gls*{datscan} images of healthy subjects present two
  ``comma-shaped'' patterns at the locations of the striatum nuclei
  (\autoref{fig:01}, A0).
However, in subjects affected by parkinsonian syndromes, the associated
  neuronal degeneration results in a lower uptake in the putamen.
Therefore, the activity registered by the corresponding \gls*{datscan} presents
  ``point-like'' patterns located at the areas of caudates (\autoref{fig:01}, A1-A3).
\begin{figure*}[ht]
\insertgraphic{width=1.0\linewidth}{figures/01-ProblemStatement}
\caption{\label{fig:01} \textbf{Problem statement.} 
On the left panel, A sub-figures show the standard classification by 
  the progression state of Parkinson's Disease: 
  A0 (normal), A1 (grade 1), A2 (grade 2), and A3 (grade 3). 
  (reproduced from \cite{center_for_drug_evaluation_and_research_fda_2011}).
On the right panel, B0 is a close-up of the template to which the subjects are registered, B1 depicts 
  one pathologic subject (grade 1) for which a corresponding \gls*{mri} scan was available, 
  and B2 is the resulting template nonlinearly mapped onto subject's space using a standard 
  image registration procedure.
Finally, C shows the \gls*{mri} scan corresponding to B1.
In blue color, the contours delineating caudates and putamens extracted from the \acrlong*{t1}.
In yellow, the corresponding contours from the template projected into the subject's space through 
  the deformation field obtained with nonlinear registration, showing the poor matching
  or the underlying anatomy.
}
\end{figure*}

\subsection{State of art}\label{sec:state-of-art}
Even though diagnosis is broadly based on experienced observers' visual
  assesment, some efforts have been made to help automate the quantification
  process and standardize the outcome (we refer the reader to
  \citep{badiavas_spect_2011} for an overview).
Automated protocols typically use a template created by averaging a healthy
  population (see \autoref{fig:01}, box B0).
Then, a number of \glspl*{roi} are defined along the resulting striatum 
  structures in the template, that will be used to quantify the uptake and characterize the
  pattern present in the image under analysis.
Therefore, the key processing step to automate quantification is the
  spatial mapping (or \emph{spatial normalization}) of the subject's image into 
  the template space.
This alignment implies an image registration process, which is challenging for two reasons.
The first caveat  is the scarce information that \gls*{datscan} images 
  retain from the underlying anatomical structures (see \autoref{fig:01}),
  also resulting from the typically low resolution of
  \gls*{datscan} images (approximately $2.5mm\times2.5mm\times2.5mm$).
Since clinical protocols usually do not include the acquisition of anatomical images
  (e.g. \gls*{t1}), in general no structural reference is available.
The second caveat is the implicit difficulty of \emph{subject-to-atlas} 
  --or, in this case, subject-to-template-- registration, due to the unpredictable 
  variability posed by severe pathologies (e.g. tumors, resected tissues, etc.).
In \gls*{datscan} quantification, the uptake pattern dramatically changes 
  with the progression of the disorder (see \autoref{fig:01}, boxes A0-A3).

The standard approach to solving the alignment problem between the subject's
  \gls*{spect} image and the \gls*{datscan} template is split in two parts.
The first step (\emph{Initialization stage}) aims at finding an affine subject-to-template mapping.
This initial affine registration linearly aligns the images to normalize for 
   head location, orientation, and head sizes.
The second step is a \emph{Refinement stage} that optimizes the local alignment of
  the uptake foci present in the image.
Some software tools have been proposed for this purpose.
For instance, \gls*{basgan} proposes a local registration scheme, clipping the 
  uptake foci and using cross-correlation between the subject's image and 
  the template as objective function.
\citeauthor{ciarmiello_weighted_2013} recently proposed \gls*{brainnest}, which
  generates a \emph{weighted} image with enhanced putaminal uptake in
  subjects that present an advanced grade of disease.
\cbstart
Currently, our \emph{in-house} approach is implemented adapting the nonlinear registration
  tools included with \gls*{spm}.
We use the \emph{spatial normalization} module configured to impose a strong regularization penalty,
  that needs to be manually fine-tuned.
Therefore, visual inspection of the results is mandatory in these settings.
\cbend
The assessment of aforementioned solutions has been, mainly, phantom-based
  and semi-automated \citep{skanjeti_assessing_2013}.
To our knowledge, no cross-comparison study has been published so far.


\cbstart
\subsection{Our contribution}\label{sec:contribution}
In this work, we propose an evaluation framework implemented in Python for the
  spatial normalization of \gls*{datscan} images required by absolute quantification
  protocols.
Quantitative results of the assessment are reported for the synthetic
  phantoms and simulated \gls*{spect} images described in \autoref{sec:data}.
Finally, the aptness of the tools is also tested on real datasets,
  assessed by visual inspection (see \autoref{sec:results_real}).
In order to demonstrate the utility of the evaluation framework, we compare 
  two methods:
(1) an implementation of our \emph{in-house} method, which uses \gls*{ibr};
(2) \gls*{pysbr}, a new shape-based registration tool and
  work in progress described in \autoref{sec:pysbr}.
\cbend
The evaluation framework, \gls*{pysbr}, one template,
  and a representative subset of the presented data are publicly 
  available in \url{https://github.com/oesteban/PySBR}.

\section{Materials}\label{sec:data}

Due to the absence of ground truth data to assess the outcomes, validation and evaluation 
  of signal processing methods are notoriously difficult tasks.
\cbstart
In order to produce a reliable benchmarking framework, we use two 
  \gls*{datscan} databases generated from real \gls*{mri} scans.
These data are used as ``ground truth'' to assess the functional operation
  of the evaluated tools.
The first database comprises synthetic phantom images (\autoref{sec:data_synthetic})
  that mimick real \glspl*{datscan}.
The second group are the results of a comprehensive simulation of the acquisition 
  process as described in \autoref{sec:data_simulated}, obtaining realistic images.
\fixme{should we mention here the real datasets as well?}
\cbend

\subsection{Synthetic phantoms database} %
\label{sec:data_synthetic}
We selected randomly 60 source \gls*{t1} images from a publicly available
  \gls*{mri} database \citep{hill_ixi_2006} of healthy volunteers.
The striatal cavities and a brain mask were segmented from each dataset
  using \emph{Freesurfer} \citep{fischl_freesurfer_2012}.
Then, based on the extracted \glspl*{roi} corresponding to the caudate and putamen
  nuclei, we generated very simplistic \gls*{spect}-like images mimicking disease in
  four stages of severity: healthy, and the three remaining stages.
\cbstart
The procedure involves some mathematical morphology operations to reproduce the
  shape that basal ganglia present in \glspl*{datscan}, emmiting signal with
  normaly distributed noise.
For further details, please see \autoref{supl:synth_phantom_wf}
\cbend
Therefore, a total of four datasets (healthy as ``grade 0'',
  and diseased as ``grades 1, 2, and 3'') were generated from each \gls*{mri}.
Finally, all the synthetic phantoms were downsampled and cropped to the characteristic
  grid and \acrlong*{fov} of typical \glspl*{datscan} ($100\times100\times100$ matrix,
  $2.56mm\times2.56mm\times2.56mm$ voxel size).

\subsection{Simulated database} %
\label{sec:data_simulated}
\cbstart
A total of 23 datasets were generated using real \gls*{mri} data as
  anatomical model.
\cbend
\Gls*{spect} projections were simulated for a Siemens E-CAM gamma
  camera equipped with a \acrlong*{lehr} parallel-hole collimator.
The acquisition parameters were selected accordingly to routinary conditions.
Thus, a total of 128 projections were obtained over 360{\degree} using a
  bin size of $3.9\times3.9mm^{2}$, a radius of rotation of $14.5cm$, and
  a $15\%$ energy window.
Subject-specific projections were obtained by using an in-house version
  \citep{crespo_quantification_2008} of the \emph{SimSET} software
  \citep{harrison_preliminary_1993}, which adapts the original code
  to $^{123}$I-labelled radioligands.
This version uses different approaches to model the interactions of
  the high-energy photons within the collimator and the detector depending
  on the energy of the simulated photons 
  \citep{cot_study_2004, cot_modeling_2006}.
To generate the activity maps, 23 \gls*{mri} images acquired from healthy
  volunteers were segmented into \gls*{gm}, \gls*{wm}, and \gls*{csf}
  using \gls*{spm}.
A different striatal to background ratio was considered for each individual
  (i.e. each subject was assigned a disease stage randomly).
Non-specific uptake was considered to be the same for \gls*{gm} and \gls*{wm},
  whereas the activity from the \gls*{csf} regions was assumed to be zero.
Specific uptake was considered by the segmentation of the striatal cavities
  using \emph{FIRST} \citep{patenaude_bayesian_2011},
  and selecting the caudate and putamen nuclei of each subject.
Attenuation maps were generated from a normalized CT image that was
  segmented into water and bone using a threshold value.
Non-uniformity of the attenuation maps was obtained by setting the
  corresponding attenuation coefficients for brain tissue and bone
  depending on the energy of the simulated photons.
Projections were filtered with a 2D-Butterworth filter ($0.64cm^{-1}$,
  order 5). Finally, reconstruction was performed by using filtered 
  back-projection algorithm with a ramp filter.

\subsection{Real datasets} %
\label{sec:data_real}
\fixme{how many datasets?}
\Gls*{spect} acquisition was undertaken $4h$
  after injection  of $185$\textit{MBq} of $^{123}$I-ioflupane
  (\Gls*{datscan}, GE Healthcare, Eindhoven, The Netherlands),
  with a  double-head gamma camera (E-CAM, Siemens Medical System,
  Erlangen, Germany) fitted with low energy, high-resolution collimators.
The parameters of acquisition were: radius of rotation of $15cm$,
  circular orbit; zoom $1.23$; $128\times128$
  pixels matrix size, $3.9\times3.9mm^{2}$ pixel size;
  128 projections, $40s$ per projection; 
  $15\%$ energy window.
At least 2 million counts per acquisition were obtained in all cases.
Reconstruction was performed as described for simulated studies.

\subsection{Templates} %
\label{sec:data_templates}
\cbstart
In standard quantification of \gls*{datscan} images, a \emph{template} is understood as
  a reference image resulting from averaging a certain number of \gls*{spect}
  images selected to build the template.
For the evaluation framework proposed here, we include some additional information
  under the same concept, such as the ``Search-\gls*{roi}'' --a mask clipping inside
  the striatal nuclei-- and a set of landmarks that will be defined in \autoref{sec:pysbr}.
\cbend
Two templates were built using subsets of cases of the simulated and real
  databases (\autoref{sec:data_simulated} and \autoref{sec:data_real}), respectively.
For the experiments using synthetic phantoms (\autoref{sec:data_synthetic}),
  we used the template generated for the simulated database.
The simulated template was created averaging 14 ``healthy'' subjects selected randomly from
  the numerical simulations.
All the subjects were spatially normalized to the \gls*{t1} template distributed with
  \gls*{spm}2.
This normalization step was validated by visual inspection.
The normalized \glspl*{mri} were resampled to have a matrix size of
  $111\times134\times102$ and a voxel size of $2.0mm\times2.0mm\times2.0mm$.
\Gls*{spect} studies were simulated (as described in \autoref{sec:data_simulated})
  on the native \gls*{mri} space, spatially normalized applying the transform
  corresponding to the subject, and intensity was normalized to a
  uniform number of counts.
The final template was obtained by averaging the normalized \gls*{spect} studies
  and filtering the result with a Gaussian filter of $8.0mm$ (isotropic)
  full width at half maximum.
The real template was obtained with a very similar procedure. 
First, we selected 10 real datasets from a healthy volunteers database, acquired
  with a \gls*{t1} reference and a coupled \gls*{spect} image obtained as
  described in \autoref{sec:data_real}.
Second, the \gls*{mri} were normalized to the \gls*{t1} template of \gls*{spm}2.
Third, the \gls*{spect} studies were rigidly registered to their corresponding \gls*{t1}, 
  and then the normalization transform was applied.
Finally, the resampling, the normalization of \gls*{spect} counts, and the final
  smoothing filter were applied as explained for the simulated template.

\section{Methods}
\label{sec:methods}

\subsection{Nipype-based evaluation workflow}
\label{sec:evaluation}
\paragraph{Software design}\label{sec:workflow}
The evaluation framework comprises the following parts: 
\begin{enumerate}
\emitem{Initialization stage}: a fast registration process shared by the
  spatial normalization packages under comparison. This module is described
  in \autoref{sec:meth_initialization}.
\emitem{Refinement stage}: where the tools under evaluation can be inserted.
  The workflow relased can execute this step either by using \gls*{pysbr} 
  or \gls*{ants} at the evaluators digression. This module is described
  in \autoref{sec:meth_refinement}.
\emitem{\Glspl{roi} mapping}: use the displacement field resulting from
  the previous step to project the template's anatomical \glspl*{roi} to
  subject's space.
\emitem{Benchmarking and reporting}: overlap scores (Dice indices) are
  computed for the four \glspl*{roi} (left and right caudates and putamens)
  and a volume-weighted average of them. Also, plots presented in
  \autoref{fig:results} are automatically generated.
\emitem{Synthetic phantom generator}: an additional workflow is released
  implementing the procedure described in \autoref{sec:data_synthetic} to
  generate the synthetic phantoms database.
\end{enumerate}
We selected \gls*{nipype} to build up the system for its unique properties for
  the construction of processing pipelines for neuroimage (i.e. flexibility,
  comprehensive interfaces to widely-used toolboxes, efficient batch processing,
  track of provenance, etc.).
Particularly, the use of \gls*{nipype} helps ensure the repeatability of
  experiments (provided with open-data) and enables fast algorithm prototyping
  as we illustrate with the \gls*{pysbr} tool.
The workflow graph (automatically generated by \gls*{nipype}) is depicted in 
  \autoref{fig:02-new}.

\begin{figure*}[!ht]
  \centering
    \insertgraphic{width=\linewidth}{figures/04-nipype-workflow}
  \caption{\label{fig:02-new}
  \textbf{Proposed evaluation workflow}. Automatically generated graph 
  representing all the software elements (as nodes) necessary
  for the assessment of \gls*{datscan} quantification tools.}
\end{figure*}


\paragraph{Visual inspection} %
Real \glspl*{datscan} can only be validated visually due to the 
  absence of structural \gls*{mri} data to precisely extract
  the evaluation \glspl*{roi}.
The database described in \autoref{sec:data_real}
  is a subset from a pool of cases that were previously aligned with an
  affine registration step, but rejected for quantification due to the
  visual mismatch with the template.
Additionally, as an exception, one real dataset was acquired with its
  corresponding \gls*{mri}.
This unique dataset allowed us to quantitatively evaluate \gls*{pysbr} on one real case,
  implicitly assuming that the necessary image registration from the 
  \gls*{spect} image to the reference \gls*{t1} is error-free 
  (or, at least, significantly smaller than the \gls*{spect}-template misregistration).
This case is used to produce the visual results illustrated in
  \autoref{fig:real_dataset_result}.

\paragraph{Implementation} %
The evaluation workflow implements the assessment of the objective tool on the three different
  databases presented in \autoref{sec:data}. 
It creates the necessary registration workflows to perform the template-subject 
  registration of one subject from a large database.
Evaluation indices are reported for \gls*{pysbr}, \gls*{ibr} and the affine initialization,
  storing them in \texttt{.csv} (comma-separated values) files. 
It also offers a simple interface (see \autoref{list:fullev}).
\begin{lstlisting}[float,caption={\label{list:fullev}Running the evaluation pipeline}]
ev = sbrwfs.evaluation()
ev.inputs.inputnode.template = 'myTemplate'
ev.inputs.inputnode.data_dir = '/path/to/database'
ev.inputs.inputnode.subject_id = 'example_id_001'
ev.inputs.inputnode.grade = 0
ev.run()
\end{lstlisting}
\noindent 
This workflow can be configured to operate in batch mode over a
  set of cases, using the \emph{Iterables} feature from \gls*{nipype}.

\subsection{Initialization stage}
\label{sec:meth_initialization}
The initial step performs a preliminary affine approximation of the selected
  template to the subject's image as explained in \autoref{sec:intro}.
Therefore, we configured \gls*{ants} to operate with a multi-resolution scheme
  of two levels, as described in \autoref{Tab:01} (levels 0 and 1)..
The resulting affine matrix is then applied to project information defined on
  the template's space into the \gls*{datscan} image.

\subsection{Refinement stage}\label{sec:meth_refinement}
As introduced before, this stage is defined with a certain interface where the
  tool under evaluation can be plugged-in.
To demonstrate the fast prototyping and \emph{functional-test driven} development
  we promote, we provide two built-in refinement tools.
On one side, our aim was to include any of the normalization tools introduced in
  \autoref{sec:intro}.
Unfortunately, at the time of writing, the \gls*{basgan} software was not available
  on the URLs provided in \cite{calvini_basal_2007,skanjeti_assessing_2013}.
Although we were able to download \gls*{brainnest} binaries, as they are compiled 
  only for Windows platforms we could not integrate it in our pipeline.
Additionally, both tools are not conceived to be easily integrated into third-party
  workflows due to their design as standalone and visual programs.
Therefore, we propose as reference methodology the \acrfull*{ibr}.
Even though these spatial normalization approaches are usually performed with \gls*{spm}
  (to be consistent with the template generation methodology), we propose here 
  a normalization scheme based on the comprehensive and fairly new tool called \gls*{ants}.
The full description of the settings for \gls*{ants} are provided in \autoref{Tab:01}.

\begin{table*}[!t]
  \processtable{\textbf{Configuration of Initialization and Refinement stages using \gls*{ibr}}.
                Registration levels and corresponding settings of \gls*{ants} are summarized.
                Levels 0 and 1 implement to the \emph{Initialization stage}, common
                for all the tools evaluated.
                The remaining two levels correspond describe the \emph{Refinement stage} details
                of the nonlinear \gls*{ibr} algorithm.\label{Tab:01}}%
    {\begin{tabular}{lp{2cm}p{11cm}}
        Level & Sub-sampling                   & Settings     \\
        \hline 
        \hline  \\[-0.8em]
        0     & Regular$^{*}$ \mbox{($4\times4\times4$)}    & Mutual information metric (``Mattes'' option in \gls*{ants}) with joint histogram of 64 bins,
                                                       rigid (6 parameters) transformation model, $\sigma_s=8.0mm$, $w=\left[25,95\right]$.
                                                       Initialized with a translation mapping geometrical centers of template and subject's images.\\
        \hline \\[-0.8em]
        1     & Regular$^{*}$ \mbox{($2\times2\times2$)} & ``Mattes'' (64 bins), affine transform (12 parameters), $\sigma_s=4.0mm$, $w=\left[25,95\right]$. \\
        \hline \\[-0.8em]
        2     & Mask$^{**}$ \mbox{($1\times1\times1$)}   & Cross-correlation metric (``CC'' option in \gls*{ants}) with 6 pixels radius,
                                                    \glsreset{syn}\gls*{syn} transformation model, $w=\left[15,100\right]$, $\tau_g = 2.0$, $\sigma^2_u = 6.0mm$, $\sigma^2_f = 10.0mm$. \\
        \hline \\[-0.8em]
        3     & Mask$^{**}$ \mbox{($1\times1\times1$)}   & ``CC'' (1 pixel radius), \gls*{syn}, $w=\left[15,100\right]$, $\tau_g = 0.75$, $\sigma^2_u = 4.0mm$,
                                                    $\sigma^2_f = 6.0mm$. \\
        \hline \\[-0.8em]
        All   & -                                 & Images preprocessed with: histogram matching filter, pre-smoothing gaussian filter of standard deviation $\sigma_s$,
                                                    winsorised intensity with window at $w = \left[ w_{low}, w_{high}\right]$ percentiles. \\
    \end{tabular}}%
    { where $\tau_g$ is gradient step, $\sigma^2_u$ is the variance of the update field, and $\sigma^2_f$ is the variance of the total field. \\
      ($^{*}$) ``Regular'' subsampling stands for decimation by integer factors of the image's grid. \\
      ($^{**}$) ``Mask'' indicates that a ``Regular'' subsampling was used, with ``Search-ROI'' as mask on the fixed image data.%
    }%
\end{table*}


\section{PySBR: Shape-based registration prototype}\label{sec:pysbr}
\subsection{Methods}
A flowchart to overview \gls*{pysbr} is presented in \autoref{fig:02},
  and described hereafter.
After the \emph{Initialization stage}, the pre-cached features in the template space are projected
  onto the subject's coordinate system and fed to a shape-matching function.
On the subject's branch, the corresponding features are detected.
Then, the shape-matching function evaluates the similarity between features.
Finally, we apply a kernel transform to propagate the local displacements of
  the features along the coordinate system of the template, obtaining the final
  projection of the template onto the subject's image.

% % % % % % % % % % % % % % % % % % % % % % % % % % % % % % %
% TODO: remove the figure environment
% % % % % % % % % % % % % % % % % % % % % % % % % % % % % % %
\begin{figure*}[!ht]
\centering
  \begin{subfigure}[bt]{0.65\textwidth}
    \insertgraphic{width=\linewidth}{figures/02-RegistrationFlowDiagram}
  \end{subfigure}
  \hfill
  \begin{minipage}[bt]{0.30\textwidth}
  \caption{\label{fig:02}
  \textbf{Internal design of \gls*{pysbr}}. \Gls*{pysbr} follows a rather standard
  registration pattern for spatial normalization, replacing the intensity-based
  metric by a shape-based one. In the \emph{Initialization} step (yellow box),
  a global alignment is found, and the shape-descriptors from the template
  are projected onto subject's space. Also, image data is masked outside the
  so-called ``Search-\gls*{roi}'' on which the features are sought. \\
  Finally, \gls*{pysbr} iteratively seeks for the features mapping in
  shape-features space. The mapping found can be propagated to the full
  image domain using a kernel transform.
  }
  \end{minipage}
\end{figure*}


\paragraph{Shape features extraction}
\label{sec:shape_features}

\cbstart
The shape features used for registration are currently restricted to two landmark sets:
\cbend
On one hand, landmarks on the brain surface are used as boundary condition to limit the nonlinear 
  transformation of the test image. 
On the other hand, landmarks are extracted from the activation areas. 
\cbstart
The \emph{Initialization stage} is intended to solve the registration problem of matching brain location
  and size between subject's image and template.
Then, we define a number of landmarks (\lm{static}) on the brain's pial surface, which are assumed
  to be correctly aligned in both images after initialization.
These \lm{static} are set to disable global changes on scale and location of the brain in the following
  nonlinear registration performed in the \emph{Refinement stage}.
\cbend
To this end, \lm{static} are uniformly located on the brain surface and present
  zero displacement. The \lm{static} are located as described in \autoref{alg:lstatic}.

\begin{figure}
\centering
\insertgraphic{width=0.75\linewidth}{figures/03-Segflow}
\caption{ \label{fig:featlmextract} Workflow to extract landmarks related to the activation areas. 
}
\end{figure}

Because at the shape feature extraction step the test image is already linearly registered to the template, 
  the activation spot search area can be restricted with the search-\gls*{roi} mask obtained from the template.
To obtain the \emph{mapping} landmarks (along the activation areas), the procedure
  outlined in Figure \ref{fig:featlmextract} is used.
\cbstart
As a result, the centers $c$ of the Caudates are estimated as the \emph{primary landmarks} of the activation area, the point 
    the middle between these two primary landmarks as the \emph{center landmark}, and a set of secondary landmarks. 
For a detailed description of the procedure see \autoref{alg:lspots}.
\cbend

\cbstart
In summary, by this shape feature extraction step three landmark sets are obtained: the static landmarks \lm{static} on the 
\cbend
  brain surface, the landmarks related to the left \lm{*,l} and right \lm{*,l} activation area, 
  and the  landmark centered between the two activation centers \lm{center} that are then used for the 
  feature based nonlinear registration described below.

\paragraph{Registration methodology}
\label{sec:registration}

\cbstart
In order to correct for the anatomical variability between patients with regard to location and orientation of the segmented nuclei, we
  perform a registration step in spherical coordinates.
\cbend
We extract the boundary of each label extracted from the \gls*{spect} images. The cartesian pixel coordinates are
  transformed to a spherical coordinates system centered at the primary 
  landmark, obtaining the ``$(r,\theta,\phi)$-plot'' or 3D ``boundary-profile''
  of the objects \citep{davies_computer_2012}.
This profile shows a flat surface for point-like shapes (typically
  present in patients with grade 2 and 3) and a characteristic distribution for
  comma-like shapes. Interpolating the profile in a regular grid in $(\theta,\phi)$, 
  we obtain a 2D gray-scale map, where the intensity is associated to $r$.
Registration is then performed to the corresponding boundary-profiles cached from
  template data. This registration is restricted to translations in the
  $(\theta,\phi)$ space, what is finally translated into small corrections of
  the relative angular missalignment of each nuclei (i.e. correction of
  azimuth and elevation of the putaminal tails \emph{seen} from the caudate heads).

\paragraph{Deformation model}
\label{sec:deformation}
Once the landmarks have been extracted and mapped, a dense deformation field of the image
  is generated, using \gls*{ebs}. \gls*{ebs} makes use the Navier 
  equilibrium \acrlongpl*{pde} for an homogeneous isotropic elastic body
  subjected to forces \citep{davis_physics-based_1997}:
  \begin{equation}
    \label{eq:ebs_pde}
    \mu \nabla^2 \vect{u}(\vect{x}) + (\mu+\lambda) \nabla \left[ \nabla \cdot \vect{u}(\vect{x}) \right] = \vect{f}(\vect{x}),
  \end{equation}
  where $\vect{u}(\vect{x})$ is the displacement of a point within the body
  from the original (unloaded) position $\vect{x}$; $\nabla^2$ and $\nabla$
  are the Laplacian and gradient, repectively; $\nabla \cdot \vect{u}(\vect{x})$
  is de divergence of the displacement field; $\vect{f}(\vect{x})$ is the force
  field defined by the landmarks mapping, and $\mu$ and $\lambda$ are the Lam\'e
  coefficients, which describe the physical properties of the elastic material.
Therefore, the $\vect{u}(\vect{x})$ found is the solution to our problem and
  can be used to project the quantification \glspl*{roi} to the subject's image.

\subsection{Software Description}
\label{sec:software}
\paragraph{Design considerations}
\label{sec:design}

\Gls*{pysbr} is a medical image application software, created
  in \emph{Python} and \emph{C++} with the following design aspects:

\begin{itemize}
  \emitem{Modularity} Using \gls*{nipype}, \gls*{pysbr} provides
    several interfaces that encapsulate the unitary processing steps
    and workflows that make use of these interfaces.

  \emitem{Reusability} Along with the main parts of the software,
    comprehensive workflows for evaluation and demonstration of the 
    software capabilities are provided.

  \emitem{Compatibility} Using \emph{Python} and \gls*{nipype}, the
    software is intended to offer high interoperability with third party
      packages.
\end{itemize}

Aside from \gls*{nipype}, \Gls*{pysbr} also relies on \emph{numpy},
  \emph{nibabel}, and \emph{scipy} as main \emph{Python} dependencies.
We also demonstrate the interoperability of \emph{Python} and \gls*{itk},
  using the \gls*{itk}'s \emph{python wrapping} to integrate the
  transformation model. %

\Gls*{pysbr} consists of two logical parts.
First, the template data that is used as the basis for the data normalization.
Second, the software implementation that in itself is split into a plain
  python code, and an extension implemented in C++.
The python code is then structured into three major design elements (python modules):
  \emph{Tools}, \emph{Interfaces}, and \emph{Workflows}.
Within the \texttt{tools} module we implement auxiliary functions,
    the core of shape evaluation, and registration routines.
It also provides the interface to the C++ extensions
    (implemented in this language for their high computational cost).
The second abstraction level is offered by the \texttt{interfaces} module, and
  the third level (end user applications) are encapsulated by the \texttt{workflows} module.
Both \emph{Interfaces} and \emph{Workflows} levels strictly follow the
  \gls*{nipype} specifications, and use the so-called \emph{Workflow Engine}
  \citep{gorgolewski_nipype:_2011}.
Consequently, the different processing units described in \autoref{fig:01} 
  are implemented as \emph{Interfaces}. 
They are fully functional to be used as stand-alone modules, or 
  they can be \emph{Nodes} of higher level \emph{Workflows}.
  
In \autoref{supl:implementation_details} we describe the main \emph{Interfaces}
  available to perform the unitary processing steps of \gls*{pysbr}.
In \autoref{supl:user_interface} we introduce the main \emph{Workflows} that are
  distributed under \gls*{pysbr}, especially the main application interface
  for normalization of \gls*{datscan} studies in standardized
  quantification of \gls*{spect} images.

\section{Results and discussion}
\label{sec:results}

\subsection{Cross-validation}
\label{sec:results_test}
We use the evaluation workflow on both groups of datasets
  for which we have a gold-standard (synthetic phantoms and simulated \gls*{spect}).
Numerical results from this evaluation are presented in \autoref{fig:results}.

%For the synthetic phantoms, 
%In the case of simulated \gls*{spect} datasets, 
%Fortunately, the caching and parallelizing features of \gls*{nipype} ease
%  this burden significantly.

\subsubsection{Evaluation on synthetic phantoms}
\label{sec:results_phantoms}
A total of 240 images, embodying the 4 different stages
  of Parkinson's Disease for the 60 selected subjects
  (\autoref{sec:data_synthetic}), were used to evaluate
  the two methodologies (480 experiments in total).
Quantitative results, grouped by grade of the stage of
  syndrome and spatial normalization methodology, 
  are given in \autoref{fig:results_a}.

\glsreset{ibr}
With the term \gls*{ibr}, we identify the spatial normalization
  workflow implemented with \gls*{ants}.
Results show that \gls*{ibr} behaves consistently,
  but offers very small improvements of overlap with respect to the
  initial mapping.
From \autoref{fig:results_a} we can interpret that there is not 
  a meaningful improvement on applying \gls*{ibr} solutions to 
  the refinement problem.
Moreover, subplots presented for ``grades'' 2 and 3 demonstrate the
  impact of the progression of the syndromes described in 
  \autoref{sec:intro}, i.e. that a number of negative samples fall 
   statistically beyond the normal span of overlaps (marked with red ``+'' symbols).
For the majority of the datasets only small improvements ($\approx1.00\%$)
  with respect to the initial overlap could be achieved.
Nonetheless, the number of statistical outliers with large degradations
  (negative change of overlap) grows notably with the advance of the state of
  pathology under study.
The effect is particularly clear for ``grade 3'', for which the outliers are enough
  to deviate the estimated mean increments (marked with stars) with respect to their
  corresponding median.
Although the appearance of negative ``outliers'' is boosted with the advance
  of pathology, \gls*{ibr} maintains an acceptable level of accuracy as it keeps
  a positive (although small) positive median increment for all the studied \glspl*{roi}.
  
Regarding the performance of \gls*{pysbr}, experiments show that the application of the tool
  did not result in a reliable enhancement of the alignment.
In fact, the estimated medians of each \gls*{roi} are always negative: i. e. in average
  the normalization achieved with the initialization was not improved.
Using the evaluation tool, we detected serious problems with the automatic 
  estimation of the landmarks associated with the caudate and putamen.
As it is clearly depicted in \autoref{fig:results_a}, this is especially true for the left-side
  structures, which always scored much worse than their right-side counterparts.
Notwithstanding, in some scarce cases notable improvements of the overlap measure could be achieved
  --  when the landmarks could be located at their right positions.
We interpret these results as promising for \gls*{pysbr}, as it shows that the proposed registration 
  approach works well if the primary activation area landmarks can be detected properly.

\subsubsection{Evaluation on simulated \gls*{spect} datasets}
\label{sec:results_simulated}
The equivalent box plot for this database is presented in \autoref{fig:results_b},
  depicting very similar conclusions.
This second database presents increased difficulties in terms of signal-to-noise
  ratio, resolution, and other artifacts related to the simulated acquisition
  procedure.
These difficulties show a certain impact, especially on the robustness of the
  methodologies under analysis.
\Gls*{ibr} shows a similar performance as above, but in this second experiment the addition
  of this refinement step does not improve the initial mapping at all.
\Gls*{pysbr} also suffers from the additional difficulties, but the impact is
  also limited.

\begin{figure*}[ht]
\centering 
\begin{subfigure}[b]{0.61\linewidth}
\insertgraphic{width=1.0\linewidth}{figures/05-results-phantoms}
\caption{\label{fig:results_a}
Synthetic phantoms.
}
\end{subfigure}
\hfill
\begin{subfigure}[b]{0.33\linewidth}
\insertgraphic{width=1.0\linewidth}{figures/05-results-simulated}
\caption{\label{fig:results_b}
Simulated \gls*{spect} datasets.
}
\end{subfigure}

\caption{\label{fig:results}
\textbf{Overlap improvement after refinement stage.}
The box plots represent the Dice index increments with respect to the baseline overlap
  indices set by the initial mapping. One box is presented for each \gls*{roi}:
  left caudate (LCAU), left putamen (LPUT), right caudate (RCAU), right putamen (RPUT),
  and the weighted average of them (``All''). Positive values represent improvement,
  negative values represent deterioration.
Presented plots are standard: boxes extend from the lower to upper quartile values
  of the data, with a line at the median. The whiskers extend from the box to show
  the range of the data. Flier points (red ``+'' symbols) are those past the end of
  the whiskers.
}
\end{figure*}

\subsection{Real datasets evaluation}
\label{sec:results_real}
Visual assessment by two experts is performed on the real datasets,
  validating the performance of the tool in all the cases.
In \autoref{fig:real_dataset_result} we reproduce the results obtained
  for the real dataset with paired anatomical reference (a \gls*{t1}),
  and both tools under analysis.
The benefits of shape-based registration were more clear on real \gls*{datscan}
  images, for two reasons.
First, the landmarks for the template computed from real datasets were 
  better located.
Second, the performance of the \gls*{ibr} approach was clearly worse on
  real datasets than that for the previous experiments.

\begin{figure}
\centering 
\insertgraphic{width=1.0\linewidth}{figures/06-results-mri-pysbr}
\caption{ \label{fig:real_dataset_result} 
  \textbf{Side-by-side view of the results obtained for a real dataset}.
  Left figure shows the standard \gls*{ibr} using \gls*{ants}, reproduced from
  \autoref{fig:01}, and the corresponding result achieved with \gls*{pysbr}
  (right). Blue contours represent the ground-truth. Yellow contours represent
  the result obtained by the two tools under evaluation.
}
\end{figure}

\glsreset{nipype}
\glsreset{pysbr}

\section{Conclusion}
\label{sec:conclusion}
Within the framework of absolute quantification of \gls*{spect}
  imaging studies for the evaluation and monitoring of movement
  disorders, we propose a workflow implemented with \gls*{nipype}
  to evaluate current spatial normalization techniques that are
  crucial for the described applications.
The evaluation workflow demonstrates the aptness of Python 
  and \gls*{nipype} for the rapid development of new methodologies
  and algorithms in image processing, while meeting
  all the requirements to enable reproducible research.

Spatial normalization of \gls*{spect} imaging studies is
  usually performed by means of intensity-based registration
  techniques in two steps.
First step involves the affine mapping of the subject and the
  reference template.
The second, is a refinement using a nonlinear transformation.
The evaluation framework enables to show 
  how intensity-based normalization often fails
  for advanced cases of Parkinson's Disease.

We present \gls*{pysbr}, an alternative tool for the
  spatial normalization using shape-features.
\Gls*{pysbr} is also implemented in \emph{Python} using
  \gls*{nipype}, and a number of publicly available tools
  for computational science and image processing
  (\emph{numpy}, \emph{scipy}, and \emph{nibabel}),
  and data management and visualization (\emph{pandas},
  \emph{matplotlib}).
  
Quantitative and qualitative results illustrate that registration approach
  of \gls*{pysbr} is promising, but requires a significant improvement
  in landmark detection for robustness.
The location of these landmarks and the correct identification of
  shapes is not trivial in \gls*{spect} images for a number of reasons,
  for instance the low resolution and the loose representation of
  underlying anatomy in \glspl*{datscan}.
In addition, manually placing the landmarks in the anatomical reference of the 
  templates could prove favorable for the template matching procedure.
 
The code is publicly released along with this manuscript and
  a number of datasets to help replicate our experiments and we
  invite the community to collaborate improving and testing
  the tool.
Additionally, \gls*{pysbr} will be used in clinical studies 
  at the Biophysics and Bioengineering Unit (Physiological 
  Sciences Department I, School of Medicine, University of 
  Barcelona, Barcelona, Spain), what ensures the
  continuity of support for the tool and a comprehensive
  testing for quality assessment.

\section*{Information Sharing Statement}
\label{sec:iss}

\begin{itemize}
\emitem{Availability} \gls*{pysbr} source code and datasets are
  publicly available at \url{https://github.com/oesteban/PySBR}.

\emitem{Requirements and specifications} A list of requirements and
  specifications is published in the main website, along with
  the necessary installation notes.

\emitem{Licence} This software is released under BSD license.
  A copy of the license is distributed along with \gls*{pysbr}.
  A number of the datasets presented in this work is also
  available under a Creative Commons licenses.
\end{itemize}

\section*{Conflict-of-Interest Statement}
The authors declare that the research was conducted in the absence of any commercial or financial relationships 
that could be construed as a potential conflict of interest.

\section*{Author Contributions}
OE designed the registration methodology, implemented the \gls*{nipype}-based code
  as well as the shape-based registration tool, performed all the presented experiments
  and lead the manuscript writing.
GW developed the activation spot segmentation and landmark extraction
  and worked on tuning the according parameters.
ANB and BMF generated the templates and wrote the simulated and real data
  descriptions in the manuscript.
MJLC, XS and AS contributed with the application background and participated
  in the analysis and interpretation of results.
All authors read, contributed to, and approved the manuscript.

\section*{Acknowledgement}
The authors thank Judith Gallego from the Biophysics and Bioengineering Unit, Physiological Sciences Department I,
School of Medicine, University of Barcelona, Barcelona, Spain, for the simulated \gls*{spect} database.
\paragraph{Funding\textcolon} This work was supported in part by Multimodal Imaging tools for Neurological Diseases (MIND-t) 
  project of Biomedical Research Networking center in Bioengineering, Biomaterials and Nanomedicine (CIBER-BBN), 
  by Spain's Ministry of Science and Innovation (through SAF2009-08076, TEC2011-28972-C02-02, TEC2010-21619-C04-03,
  IPT-2010-003-300000, IPT-2012-0401-300000), by CDTI-CENIT (AMIT project), and by Fondo de Investigaciones Sanitarias (PI12-00390).
BMF was awarded a PhD fellowship (App Form – Call 07-2009) of Institute for Bioengineering of Catalonia (IBEC).

%\section*{Supplemental Data}
% enable if needed

%\begin{figure}
%\centering
%\begin{minipage}[b]{0.45\linewidth}
%\insertgraphic{width=1.0\linewidth}{figures/02-RegistrationFlowDiagram}
%\end{minipage}
%\quad
%\begin{minipage}[b]{0.45\linewidth}
%\insertgraphic{width=1.0\linewidth}{figures/04-nipype-workflow}
%\end{minipage}
%\caption{\label{fig:02}
%\textbf{Internal design of \gls*{pysbr}}.
%}
%\end{figure}


\bibliographystyle{myplainnat}
%\bibliographystyle{frontiers}
%\begin{thebibliography}{}
\bibliography{references}
%\end{thebibliography}

\newpage

\onecolumn
\appendix
\section*{Appendix}
\subsection*{Algorithms used for extraction of shape features}

\begin{myalgorithm}[Extraction of static landmarks across the surface of the brain\label{alg:lstatic}]
\begin{enumerate}
\emitem{Boundary}: the boundary of the brain mask corresponding to the template is
  extracted, simply subtracting an eroded version of the mask to the original one.
\emitem{Polar mapping}: the cartesian coordinates of the voxels contained in the boundary
  are converted to spherical coordinates, setting the origin of the system on the center
  of mass of the full brain mask.
\emitem{Sampling}: then, a regular grid is defined on $(\theta,\phi)$ with a fixed number
  of positions. This grid is used to sample the mapping obtained in the previous space.
  By default, the sampling matrix is $9\times18$, with $9^{-1}\pi \times 9^{-1}\pi$ size
  to sample the full sphere.
  Drawing the nearest neighbors to the regular $(\theta,\phi)$ grid points from the spherical
  coordinates of the boundary, we uniformly sub-sample the boundary.
\emitem{Back-mapping}: we set as \lm{static} the cartesian coordinates of the
  previously selected voxels.
\end{enumerate}
\end{myalgorithm}
\begin{myalgorithm}[Extraction of landmarks related to activation areas\label{alg:lspots}]
  \begin{enumerate}
  \emitem{Hotpoint estimation}: Within the masked area two local intensity maxima or \emph{hotspots}  are searched as seeds 
    for the segmentation of the activation areas. 
  To ensure that the two hotspots are not found in the same activation area, the image is divided into the 
    eight octants separated by the center of the image and the second hotspot is selected to be 
    from another octant then the first one. 
  \emitem{Activation area segmentation}: An intensity based region growing is used to extract the two activation 
     areas simultaneously. 
  The breaking condition for this region growing are set based on the minimum intensity of the image along the line 
    connecting the two hotspots  \hp{1,l} and \hp{1,r} and the mean intensity of the masked input area.
  As the result of this procedure two labeled regions $A_{\text{spot},*}$ are obtained. 
  Here, if the volume $V_{\text{spot},*}$ of an extracted region exceeds a user set parameter $V_{\text{max}}$, 
    the segmentation is considered to have failed and the registration method can not be applied.
  \emitem{Landmark estimation}: The center $\pc$ of the Caudates are estimated as the primary activation area landmarks.
    The procedure is run independently for the left and right Caudate:
  \begin{enumerate}
  \emitem{Thinning}: A number $n_{\text{thin}}$ of morphological thinning iterations \citep{lee_building_1994} is run on 
    $A_{\text{spot},l}$ that is based on the volume $V_{\text{spot},l}$ of the segmented areas, 
    resulting in an area $\hat{A}_{\text{spot},l}$. 
  \emitem{Scanning}: In order to find the center of the left Caudate, its following properties are considered: 
  It is located at an extreme location of the activation area, it is close to the other activation area, and 
    its intensity in the activation image is relatively high. 
  Based on this and given the boundary $B_l:= \partial A_{\text{spot},l}$, 
     the quantity $\Delta_{\pc}$ is defined that weights the distance of a point $\pc \in \hat{A}_{\text{spot},l}$ 
    with in the thinned region of the left hotspot area against its minimum distance from the right hotspot area, 
    and the relative image intensity: 
  \begin{equation}
    \Delta_{\pc} := \frac{1}{V_{\text{spot},l}} \cdot 
                   \frac{\sum_{\pa \in \hat{A}_{\text{spot},l}}(\|\pc-\pa\|) }{1 + \min_{\pr \in A_{\text{spot},r}}\|\pc-\pr\| } \cdot 
                   \frac{I(\pc)}{I(\hp{1,l})}
  \end{equation}
  \noindent 
  Then, the primary landmark \lm{1,l} is evaluated like follows: 
  \begin{equation}
    \lm{1,l} = \pc_{l} := \arg \max_{ \pc\in\hat{A}_{\text{spot},l}}\left( \min_{\pb \in B}\left(\|\pc-\pb\| \cdot  \Delta_\pb \right)\right)
  \end{equation}
  \end{enumerate}
  \noindent 
  The primary landmark \lm{1,r} within the right activation area is evaluated accordingly. 
  \emitem{Center landmark}: A third  landmark is obtained as the point 
  \begin{equation}
  \label{eq:middle}
  \lm{center} := \frac{1}{2}\left(\lm{1,l} + \lm{1,r}\right)
  \end{equation}
  in the middle between these two landmarks.
  \emitem{Secondary landmarks}: Additional per-activation landmarks can then be evaluated as follows: 
  \begin{enumerate}
  \emitem{Evaluate principal direction}:
  In order to estimate secondary landmarks, a \gls*{pca} of each region is run, using \lm{1,*} as reference point, 
    to estimate its principle eigenvector $\vec{e}_{p,*}$, and the ratio $r_{\text{spot},*}$ of the two largest 
    eigenvalues, which is used to rate the eccentricity of the activation area. 
  \emitem{Quantity of landmarks}:
  Then, given on the user set parameters minimum region volume $V_{\text{min}}$ and eigenvalue ratio $r_{\text{min},*}$, 
    and based on the segmented activation volume $V_{\text{spot},*}$  and the spot eccentricity $r_{\text{spot},*}$
    the number of additional landmarks is estimated according to
  \begin{equation}
  \label{eq:nlm}
    N^+_{lm} := \min\left(  \left\lfloor \frac{ V_{\text{spot},*}}{V_{\text{min}}}\right\rfloor,
                   \left\lfloor \frac{ r_{\text{spot},*}}{r_{\text{min}}} \right\rfloor, 2\right), 
  \end{equation}
  \noindent 
  limiting this number to two at most. 
  \emitem{Correcting the eigenvector}: Since the pointing direction of the eigenvector $\vec{e}_{p,*}$ is undetermined,
    it is set so that 
  \begin{equation}
      \vec{e}_{p,*} \cdot \sum_{\vx \in \hat{A}_{\text{spot},*}}(\vx - \lm{1,*}) > 0. 
  \end{equation}
  \emitem{Adding landmarks}: 
  These additional landmarks are then added along the direction of the principal eigenvector $\vec{e}_p$ scaled by a user 
    provided scaling factor $s$ according to 
  \begin{equation}
  \label{eq:addlm}
  \lm{k+1,*} := k * s * \vec{e}_{p,*} + \lm{1,*} \: \text{with} \: k\in[1, N^+_{lm}]\subset \mathbb{N}
  \end{equation}
  \end{enumerate}
  \end{enumerate}
\end{myalgorithm}

\newpage
\appendix
\setcounter{page}{1}
\renewcommand*{\thepage}{Suplemental Material - \arabic{page}}
\section*{Suplemental Material}

\subsection*{PySBR Implementation details}
\label{supl:implementation_details}

All the interfaces present a uniform structure (input parameters, output parameters,
  and functional body), and a uniform use (creation, input parameters setting, and
  execution calling \texttt{run()}).
This behavior is inherited from the \texttt{BaseInterface} class of \gls*{nipype}.
We summarize here the most important interfaces,
  presenting some examples of their use in \autoref{list:interfaces}.
The presented code examples make use of the three modules of 
  \gls*{pysbr}, that will be imported as given in \autoref{list:interfaces}.

\subsection*{Workflow for generation of synthetic phantoms}
\label{supl:synth_phantom_wf}
\fixme{here the workflow with some exemplary images}

\subsection*{Templates}
\label{supl:templates}
Templates are a fundamental part of the software, as they hold the reference shapes to
  perform normalization.
A template is defined by the reference \gls*{datscan} image, that
  should be named as \texttt{template.nii.gz} and a corresponding brain mask (with name
  \texttt{brainmask.nii.gz}).
Once these two files are correctly placed in a folder
  (e.g. \texttt{myTemplate/}) under the templates path (that can be moved from the
  default setup at the user's convenience), the appropriate \gls*{nipype}-style
  interface can be invoked to run the template feature extraction (see
  \autoref{list:interfaces}, lines 2-3).
On the first execution of the interface, the shape features extracted from
  the template are cached for faster retrieval in subsequent calls to the interface. 
When using \gls*{pysbr} for evaluation or quantification purposes, it is possible
  to install additional files, e.g. the corresponding \glspl*{roi} extracted
  from the anatomical reference.


\subsection*{Shape analysis interfaces}
\label{supl:shapeanalysis}

\Gls*{pysbr} provides several interfaces to perform shape-based analysis of \gls*{datscan}
  images.
The first interface is \texttt{FindPrincipalSpots} that segments the \acrlong*{dat} activity
  associated to each nuclei (comprising caudate and putamen of each hemisphere),
  as shown in \autoref{list:interfaces}, lines 6-9.
As inputs, the interface requires the subject's \gls*{datscan} image and a ``search-\gls*{roi}'' or
  bounding box.
The extraction methodologies involved in this interface are described in \autoref{sec:shape_features}.

A second interface performing shape-based registration is named \texttt{SBR}
  (\autoref{list:interfaces}, lines 12-15).
This interface takes the landmarks extracted previously and, finds
  the shape-based mapping described in \autoref{sec:registration}.
Input and output files make use of the \gls*{json} format
  for interoperability reasons.

\subsection*{Transformation model} %
Currently, only the \gls*{ebs} transform from \gls*{itk} is supported
  (see \autoref{list:interfaces}, lines 18-24).
The \texttt{EBSTransform} interface composes the initial affine transform
  with the nonlinear transform found as described in \autoref{sec:deformation}
  to project the moving inputs into reference space.
In future versions of \gls*{pysbr}, other kernel interpolators implemented in \gls*{itk} 
  will be also available.
This interface extends a base class (\texttt{BaseTransform}) that is provided to
  ease these future extensions.


\glsreset{ibr}
\subsection*{PySBR User interface}
\label{supl:user_interface}

Here, We introduce two final user
  applications released with \gls*{pysbr}: a registration workflow for
  \gls*{spect} data quantification protocols, and an evaluation framework
  to reproduce the experiments we used to validate the tool.
These two workflows are also provided as command line interfaces.
Additional workflows are released with \gls*{pysbr}, for instance
  the workflow used to generate the synthetic datasets as described
  in \autoref{sec:data_synthetic} and the refinement workflow using
  nonlinear \gls*{ibr}, implemented using \gls*{ants}.

\subsubsection*{Normalization workflow} %
The core interface for the user is offered by this workflow,
  which implements the normalization of a given \gls*{spect} 
  dataset to a selectable template.

The input value \texttt{template} must hold a valid name of an installed 
  template (see \autoref{supl:templates}). 
The input \texttt{in\_file} must be the path to an existing \gls*{spect} image.

\begin{lstlisting}[caption={\label{list:interfaces}\textbf{Execution samples.}}]
# Loading the three major elements of the python interface of PySBR
import pysbr.tools as sbrtools
import pysbr.interfaces as sbrifs
import pysbr.workflows as sbrwfs

# Running a full template-subject registration
reg = sbrwfs.normalization()
reg.inputs.inputnode.template = 'myTemplate'
reg.inputs.inputnode.in_file = 'path/to/mySubject/datscan.nii.gz'
reg.run()

# Interface wrapping template selection
tpl_gen = sbrifs.TemplateSource(name='myTemplate')
tpl_gen.run()

# Interface to detect primary spots
fspots = sbrifs.FindPrincipalSpots()
fspots.inputs.in_file = 'path/to/mySubject/datscan.nii.gz'
fspots.inputs.in_bbox = 'path/to/mySubject/searchROI.nii.gz'
res1 = fspots.run()

# Interface to shape-based registration of binary boundaries
reg = sbrifs.SBR()
reg.inputs.src_spots = 'path/to/myTemplate/landmarks.txt'
reg.inputs.trg_spots = 'path/to/mySubject/landmarks.txt'
reg.run()

# Interface connecting ITK for elastic-body splines transforms
tfm = sbrifs.EBSTransform()
tfm.inputs.in_source = 'path/to/myTemplate/template.nii.gz'
tfm.inputs.in_target = 'path/to/mySubject/datscan.nii.gz'
tfm.inputs.in_initialization = [ 'transform1.mat', 'transform2.mat' ]
tfm.inputs.in_transform = 'path/to/displacements.txt'
tfm.inputs.in_apply = [ 'path/to/myTemplate/ROI1.nii.gz', 'path/to/myTemplate/ROI2.nii.gz']
tfm.run()
\end{lstlisting}

\end{document}
